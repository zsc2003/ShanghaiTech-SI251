\documentclass[10pt]{article}
\usepackage[UTF8]{ctex}

\usepackage[utf8]{inputenc} % allow utf-8 input
\usepackage{amsthm}
\usepackage{amsmath,amscd}
\usepackage{amssymb,array}
\usepackage{amsfonts,latexsym}
\usepackage{graphicx,subfig,wrapfig}
\usepackage{times}
\usepackage{psfrag,epsfig}
\usepackage{verbatim}
\usepackage{tabularx}
\usepackage[pagebackref=true,breaklinks=true,letterpaper=true,colorlinks,bookmarks=false]{hyperref}
\usepackage{cite}
\usepackage{algorithm}
\usepackage{multirow}
\usepackage{caption}
\usepackage{algorithmic}
%\usepackage[amsmath,thmmarks]{ntheorem}
\usepackage{listings}
\usepackage{color}
\usepackage{bm}

% support llbracket and rrbracket  []
\usepackage{stmaryrd}


\newtheorem{thm}{Theorem}
\newtheorem{mydef}{Definition}

\DeclareMathOperator*{\rank}{rank}
\DeclareMathOperator*{\trace}{trace}
\DeclareMathOperator*{\acos}{acos}
\DeclareMathOperator*{\argmax}{argmax}


\renewcommand{\algorithmicrequire}{ \textbf{Input:}}
\renewcommand{\algorithmicensure}{ \textbf{Output:}}
\renewcommand{\mathbf}{\boldsymbol}
\newcommand{\mb}{\mathbf}
\newcommand{\matlab}[1]{\texttt{#1}}
\newcommand{\setname}[1]{\textsl{#1}}  
\newcommand{\Ce}{\mathbb{C}}
\newcommand{\Ee}{\mathbb{E}}
\newcommand{\Ne}{\mathbb{N}}
\newcommand{\Se}{\mathbb{S}}
\newcommand{\norm}[2]{\left\| #1 \right\|_{#2}}

\newenvironment{mfunction}[1]{
	\noindent
	\tabularx{\linewidth}{>{\ttfamily}rX}
	\hline
	\multicolumn{2}{l}{\textbf{Function \matlab{#1}}}\\
	\hline
}{\\\endtabularx}

\newcommand{\parameters}{\multicolumn{2}{l}{\textbf{Parameters}}\\}

\newcommand{\fdescription}[1]{\multicolumn{2}{p{0.96\linewidth}}{

		\textbf{Description}

		#1}\\\hline}

\newcommand{\retvalues}{\multicolumn{2}{l}{\textbf{Returned values}}\\}
\def\0{\boldsymbol{0}}
\def\b{\boldsymbol{b}}
\def\bmu{\boldsymbol{\mu}}
\def\e{\boldsymbol{e}}
\def\u{\boldsymbol{u}}
\def\x{\boldsymbol{x}}
\def\v{\boldsymbol{v}}
\def\w{\boldsymbol{w}}
\def\N{\boldsymbol{N}}
\def\X{\boldsymbol{X}}
\def\Y{\boldsymbol{Y}}
\def\A{\boldsymbol{A}}
\def\B{\boldsymbol{B}}
\def\y{\boldsymbol{y}}
\def\cX{\mathcal{X}}
\def\transpose{\top} % Vector and Matrix Transpose

%\long\def\answer#1{{\bf ANSWER:} #1}
\long\def\answer#1{}
\newcommand{\myhat}{\widehat}
\long\def\comment#1{}
\newcommand{\eg}{{e.g.,~}}
\newcommand{\ea}{{et al.~}}
\newcommand{\ie}{{i.e.,~}}

\newcommand{\db}{{\boldsymbol{d}}}
\renewcommand{\Re}{{\mathbb{R}}}
\newcommand{\Pe}{{\mathbb{P}}}

\hyphenation{MATLAB}

\usepackage[margin=1in]{geometry}

\begin{document}

\title{	SI251 - Convex Optimization, 2024 Spring\\Homework 1}
\date{Due 23:59 (CST), Mar. 27, 2024 }

\author{
    Name: \textbf{Zhou Shouchen} \\
	Student ID: 2021533042
}

\maketitle

\newpage
%===============================


\begin{enumerate}

\section{Convex sets}
\item  Please prove that the following sets are convex: 
\begin{itemize}
    \item[1)] $S=\left\{x \in \mathbf{R}^m \;|\;\mid p(t) \mid \leq 1 \text { for }|t| \leq \pi / 3\right\}$, where $p(t)=x_1 \cos t+x_2 \cos 2 t+\cdots+x_m \cos m t$. {\color{red} (5 pts)}
    \item[2)] (\textbf{Ellipsoids}) $\Big\{x|\sqrt{(x-x_c)^TP(x-x_c)}\leq r\Big\}~~~(x_c\in \mathbb{R}^n, r\in \mathbb{R}, P\succeq 0)$. {\color{red} (5 pts)}
    \item[3)] (\textbf{Symmetric positive semidefinite matrices}) $S_{+}^{n\times n}=\Big\{ P\in S^{n\times n}|P\succeq 0\Big\}$. {\color{red} (5 pts)}
    \item[4)] The set of points closer to a given point than a given set, i.e.,
    \begin{equation*}
        \Big\{x~\vert~\|x-x_0\|_2\leq\|x-y\|_2~\text{for all}~y\in S\Big\},
    \end{equation*}
    where $S\in R^n$. {\color{red} (5 pts)}
\end{itemize}

(1)




(2)



(3)



(4)





\newpage

\item {\color{red} (15 pts)} For a given norm $\|\cdot\|$ on $\mathbf{R}^n$, the dual norm, denoted $\|\cdot\|_*$, is defined as
$$
\|y\|_*=\sup_{x\in\mathbf{R}^n} \{y^T x\mid\|x\|\leq1\}.
$$ 
Show that the dual of Euclidean norm is the Euclidean mom, i.e., $\sup_{x \in \mathbf{R}^n}\{z^{T}x \;| \;\|x\|_2\leq1\}=||z||_{2}.$\\







\newpage

\item {\color{red} (15 pts)} Define a norm cone as
$$
\mathcal{C} \equiv \left\{(x, t): x \in \mathbb{R}^d, t \geq 0,\|x\| \leq t\right\} \subseteq \mathbb{R}^{d+1}
$$

Show that the norm cone is convex by using the definition of convex sets.\\








\newpage

\section{Convex functions}
\item {\color{red} (18 pts)} Let $C\subset \mathbb{R}^n$ be convex and $f:C\rightarrow R^\star$. Show that the following statements are equivalent:
\begin{itemize}
    \item[(a)] epi($f$) is convex.
    \item[(b)] For all points $x_i\in C$ and $\{\lambda_i|\lambda_i\geq0, \sum_{i=1}^n \lambda_i=1, i=1,2,\cdots,n\}$, we have
    \begin{equation*}
        f\Big(\sum\limits_{i=1}^n \lambda_ix_i\Big)\leq \sum\limits_{i=1}^n \lambda_if(x_i).
    \end{equation*}
    \item[(c)] For $\forall x,y\in C$ and $\lambda\in[0,1]$,
    \begin{equation*}
        f\Big((1-\lambda)x+\lambda y\Big)\leq(1-\lambda)f(x)+\lambda f(y).
    \end{equation*}
\end{itemize}

(a)





(b)





(c)







\newpage

\item {\color{red} (14 pts)} {Monotone Mappings. A function $\psi:\mathbf{R}^n \to \mathbf{R}^n$ is called monotone if for all $x,y \in \mathbf{dom} \psi$,
    $$(\psi(x) - \psi (y))^T (x-y) >= 0$$
Suppose $f : \mathbf{R}^n \to \mathbf{R}^n$ is a differentiable convex function. Show that its gradient $\nabla f$ is monotone. Is the convex true, i.e., is every monotone mapping the gradient of a convex function?}\\







\newpage

\item {\color{red} (18 pts)} Please determine whether the following functions are convex, concave or none of those, and give a detailed explanation for your choice.
\begin{itemize}
    \item[1)] 
     \begin{equation*}
     f_1(x_1,x_2,\cdots,x_n)=
      \begin{cases}
         &-(x_1x_2\cdots x_n)^{\frac{1}n},~~ \text{if}~~x_1,\cdots,x_n>0\\
         &~~\infty ~~~~~~~~~~~~~~~~\text{otherwise};\
       \end{cases}
     \end{equation*}
    \item[2)] $f_2(x_1,x_2)= x_1^\alpha x_2^{1-\alpha}$, where $0\leq\alpha\leq1$, on $\mathbb{R}_{++}^2$;
    \item[3)] $f_3(x,u,v)=-\log(uv-x^Tx)$ on ${\bf dom} f =\{(x,u,v)|uv>x^Tx,~~u,v>0\}$.
\end{itemize}

(1)




(2)




(3)







\end{enumerate}













\end{document}