\section{Monotone Operators and Base Splitting Schemes}
{\color{red} (35 pts)}
Proof the theorem below:
\begin{theorem}
For \(v \in \mathbb{R}^n\), the solution of the equation
\begin{equation}
u^* = (I - JW)^{-T}v
\end{equation}
is given by
\begin{equation}
u^* = v + W^T\tilde{u}^*
\end{equation}
where $I$ is the identity matrix and \(\tilde{u}^*\) is a zero of the operator splitting problem \(0 \in (F + G)(u^*)\), with operators defined as
\begin{equation}
F(\tilde{u}) = (I - W^T)(\tilde{u}), \quad G(\tilde{u}) = D\tilde{u} - v
\end{equation}
where \(D\) is a diagonal matrix defined by \(J = (I + D)^{-1}\) (where \( J_{ii} > 0 \)).
\end{theorem}
(Hint-1, please refer to Monotone Operators-note.pdf)
\par
(Hint-2, $I = (I-J W)^{-T} (I-J W)^{T}$)


\solution{}
Since $D$ is a diagonal matrix, and $J=(I+D)^{-1}$, so $J$ is also a diagonal matrix. i.e. $J=J^T$.

1. When $D_{ii}<+\infty$, i.e. $J_{ii}>0$,
\begin{equation}
\begin{aligned}
&\ u^*=(I-JW)^{-T}v \\
&\Leftrightarrow (I-W^TJ^T)u^*=v \\
&\Leftrightarrow (I-W^TJ)u^*=v \text{\ \ \ \ \ \ \ \ \ \ \ \ \ \ \ \ \ \ \ \ ($J^T=J$)} \\
&\Leftrightarrow (I-W^T(I+D)^{-1})u^*=v \text{\ \ \ \ \ \ \ \ \ ($J=(I+D)^{-1}$)} \\
&\Leftrightarrow W^{-T}(I-W^T(I+D)^{-1})u^*=W^{-T}v \\
&\Leftrightarrow W^{-T}u^*-(I+D)^{-1}u=W^{-T}v \\
&\Leftrightarrow (I+D)W^{-T}u^*-u^*=(I+D)W^{-T}v
\end{aligned}
\end{equation}

Define $\tilde{u}^*=W^{-T}u^*$, i.e. $u^*=W^T\tilde{u}^*$, put it into equation (6), then we can get that
\begin{equation}
\begin{aligned}
&\ u^*=(I-JW)^{-T}v \\
&\Leftrightarrow  (I+D)\tilde{u}^*-W^T\tilde{u}^*=(I+D)W^{-T}v \\
&\Leftrightarrow  \tilde{u}^*+D\tilde{u}^*-W^T\tilde{u}^*=(I+D)W^{-T}v
\end{aligned}
\end{equation}

From Hint-2, we have $I=(I-JW)^{-T}(I-JW)^T$, so we can get that
$$\left(I-W^TJ\right)\left(I+(I-W^TJ)^{-1}W^TJ\right)=I-W^TJ+W^TJ=I$$
i.e.
$$(I-JW)^{-T}=(I-W^TJ)^{-1}=I+(I-W^TJ)^{-1}W^TJ$$

So we can get that
\begin{align*}
u^* &= (I-JW)^{-T}v \\
&= v+(I-W^TJ)^{-1}W^TJv \\
\end{align*}
If we define $v'=(W^TJ)^{-1}v$, tand put it into the equation (7), we can get that
$$\tilde{u}^*+D\tilde{u}^*-W^T\tilde{u}^*=(I+D)W^{-T}v=(I+D)Jv'=v'$$
i.e.
\begin{align*}
(I-W^T)\tilde{u}^*+D\tilde{u}^* &= v' \\
(I-W^T)\tilde{u}^*+D\tilde{u}^* -v' &= 0 \\
F(\tilde{u}^*)+G(\tilde{u}^*) &= 0
\end{align*}
So we have proved that $0\in (F+G)(\tilde{u}^*)$.\\


2. When $J_{ii}=0$, i.e. $D_{ii}=+\infty$, we can simply take the limit $D_{ii}\to +\infty$.\\
Since that $I-W^T\succeq mI$ and $D_{ii}\geq 0$, and the operators are well-defined.\\
So we can get that operator $F$ is strongly monotoneand, and the operator $G$ is monotone.

So operator splitting techniques applied to the problem will be guaranteed to converge.