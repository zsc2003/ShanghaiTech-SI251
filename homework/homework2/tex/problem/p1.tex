\item {\color{red} (50 pts)} \textbf{Robust quadratic programming}. In the lecture, we have learned about robust linear programming as an application of second-order cone programming. Now we will consider a similar robust variation of the convex quadratic program
$$
\begin{array}{ll}
\operatorname{minimize} & (1 / 2) x^T P x+q^T x+r \\
\text { subject to } & A x \preceq b .
\end{array}
$$

For simplicity, we assume that only the matrix $P$ is subject to errors, and the other parameters $(q, r, A, b)$ are exactly known. The robust quadratic program is defined as
$$
\begin{array}{cl}
\operatorname{minimize} & \sup _{P \in \mathcal{E}}\left((1 / 2) x^T P x+q^T x+r\right) \\
\text {subject to } & A x \preceq b
\end{array}
$$
where $\mathcal{E}$ is the set of possible matrices $P$.\\
For each of the following sets $\mathcal{E}$, express the robust QP as a convex problem in a standard form (e.g., QP, QCQP, SOCP, SDP).
\begin{itemize}
    \item[(a)] A finite set of matrices: $\mathcal{E}=\left\{P_1, \ldots, P_K\right\}$, where $P_i \in S_{+}^n, i=1, \ldots, K$.
    \item[(b)] A set specified by a nominal value $P_0 \in S_{+}^n$ plus a bound on the eigenvalues of the deviation $P-P_0$:
    $$
    \mathcal{E}=\left\{P \in \mathbf{S}^n \mid-\gamma I \preceq P-P_0 \preceq \gamma I\right\}
    $$
    where $\gamma \in \mathbf{R}$ and $P_0 \in \mathbf{S}_{+}^n$.

    \item[(c)] An ellipsoid of matrices:
    $$
    \mathcal{E}=\left\{P_0+\sum_{i=1}^K P_i u_i \mid\|u\|_2 \leq 1\right\} .
    $$

    You can assume $P_i \in \mathbf{S}_{+}^n, i=0, \ldots, K$.
\end{itemize}

\solution{}
Since $\sup\limits_{P \in \mathcal{E}}\left(\dfrac{1}{2} x^T P x+q^T x+r\right)=\left(\sup\limits_{P \in \mathcal{E}}\dfrac{1}{2} x^T P x\right)+q^T x+r$, so we could need to consider the $\sup\limits_{P \in \mathcal{E}}\dfrac{1}{2} x^T P x$ part of the objective function.\\

(a) Let $$t=\sup\limits_{P\in\mathcal{E}}\dfrac{1}{2}\mathbf{x}^TP\mathbf{x}$$
i.e. $$\forall i=1,\cdots,K,\ \dfrac{1}{2}\mathbf{x}^TP_i\mathbf{x}\leq t$$
So the program can be rewriten as
\begin{align*}
    \min_{\mathbf{x},t}\ \ \ \ \ & t+q^T \mathbf{x}+r \\
    \text {subject to }\ \ \ \ \ & A \mathbf{x} \preceq b\\
    & \dfrac{1}{2}\mathbf{x}^TP_i\mathbf{x}\leq t,\ \ \ i=1, \ldots, K
\end{align*}
The objective function is linear to the variable $(\mathbf{x},t)$, and the constrains are in quadratic form.\\
So above all, the problem is a QCQP.\\

(b) $\forall P\in\mathcal{E}$, we have
$$-\gamma I \preceq P-P_0 \preceq \gamma I$$
which means that $\forall\mathbf{x}\in\mathbb{R}^n$, we have
$$\mathbf{x}^T (-\gamma I) \mathbf{x}\leq \mathbf{x}^T(P-P_0)\mathbf{x} \leq  \mathbf{x}^T(\gamma I)\mathbf{x}$$
i.e.
$$\mathbf{x}^T (P_0-\gamma I) \mathbf{x}\leq \mathbf{x}^TP\mathbf{x} \leq \mathbf{x}^T(P_0+\gamma I)\mathbf{x}$$

So the program can be rewriten as
\begin{align*}
    \min_{\mathbf{x}}\ \ \ \ \ & \dfrac{1}{2}\mathbf{x}^T(P_0+\gamma I)\mathbf{x}+q^T \mathbf{x}+r \\
    \text {subject to }\ \ \ \ \ & A \mathbf{x} \preceq b
\end{align*}
The objective function is in the quadratic form to the variable $\mathbf{x}$, and the constrains are in linear form.\\
So above all, the problem is a QP.\\

(c) We can define $y_i=x^T P_i x$, then we have:
\begin{align*}
    \sup_{P \in \mathcal{E}}\dfrac{1}{2} x^T P x &= \sup_{\|\mathbf{u}\|\leq 1}\left(\dfrac{1}{2} x^T P_0 x + \sum_{i=1}^{K}\dfrac{1}{2} u_ix^T P_i x\right) \\
    &= \dfrac{1}{2} x^T P_0 x + \dfrac{1}{2}\sup_{\|\mathbf{u}\|\leq 1}\left(\sum_{i=1}^{K} u_ix^T P_i x\right) \\
    &= \dfrac{1}{2} x^T P_0 x + \dfrac{1}{2}\sup_{\|\mathbf{u}\|\leq 1}\left(\sum_{i=1}^{K} u_iy_i \right) \text{\ \ \ \ \ \ \ } (y_i=x^T P_i x)\\
    &= \dfrac{1}{2} x^T P_0 x + \dfrac{1}{2}\sup_{\|\mathbf{u}\|\leq 1}\mathbf{u}^T\mathbf{y} \\
    &= \dfrac{1}{2} x^T P_0 x + \dfrac{1}{2}\sup_{\|\mathbf{u}\|\leq 1}\|\mathbf{u}\|_2\|\mathbf{y}\|_2 \text{\ \ \ \ \ \ \ \ \ (Cauchy Inequality)} \\
    &= \dfrac{1}{2} x^T P_0 x + \dfrac{1}{2}\|\mathbf{y}\|_2
\end{align*}
So the objective function becomes $\dfrac{1}{2} x^T P_0 x + \dfrac{1}{2}\|\mathbf{y}\|_2 + q^T x+r$.\\
Since there are no suitable programmings for existing norm in the objective function, so we can convert it into the constrains.\\
Let $u=\dfrac{1}{2}x^TP_0x, t=\dfrac{1}{2}\|y\|_2$.\\
Then the current simplified problem is that
\begin{align*}
    \min_{\mathbf{x},\mathbf{y},u,t}\ \ \ \ \ & u+t+q^T \mathbf{x}+r \\
    \text {subject to }\ \ \ \ \ & A \mathbf{x} \preceq b \\
    & t = \dfrac{1}{2}\|y\|_2 \\
    & u = \dfrac{1}{2}x^TP_0x \\
    & y_i = x^TP_ix \ \ \ \forall i=1,\cdots,K
\end{align*}
We could find that the closest form for the problem is the SOCP, but has some difference, so we need to do some conversions.\\
Since $u,t$ are seperated and independent, and the transmissibility of the inequality, to better suit SOCP, we could do the scalings, which would led to the same result as taking minimum:
\begin{align*}
    t = \dfrac{1}{2}\|y\|_2 &\Rightarrow t \geq \dfrac{1}{2}\|y\|_2\\
    u = \dfrac{1}{2}x^TP_0x &\Rightarrow u \geq \dfrac{1}{2}x^TP_0x\\
    y_i = x^TP_ix &\Rightarrow y_i \geq x^TP_ix \ \ \ \forall i=1,\cdots,K
\end{align*} 
Since $P_i \in \mathbf{S}_{+}^n, i=0, \ldots, K$, so we could do eigenvalue decomposition to each matrix, which are diagonalizable due to the symmetry.
$P_i=Q_i^{-1}\Lambda_iQ_i$, and for all eigenvalues in $\Lambda_i$ is non-negative, so we have $P_i^{\frac{1}{2}}=Q_i^{-1}\Lambda_i^{\frac{1}{2}}Q_i$.\\
And construct an inequality:
$$\left\|
\left[\begin{array}{l}
P_0^{\frac{1}{2}}x \\
u-\dfrac{1}{2}
\end{array}\right]\right\|_2
\leq
u+\dfrac{1}{2}
$$
If we square to the both side, we can get that
\begin{align*}
    \|P_0^{\frac{1}{2}}x\|_2^2+(u-\dfrac{1}{2})^2 &\leq (u+\dfrac{1}{2})^2\\
    x^TP_0x+u^2-u+\dfrac{1}{4} &\leq u^2+u+\dfrac{1}{4} \\
    \dfrac{1}{2}x^TP_0x &\leq u
\end{align*}
Similarly, in the exactly same way, we can construct
$$\left\|
\left[\begin{array}{l}
P_i^{\frac{1}{2}}x \\
y_i-\dfrac{1}{4}
\end{array}\right]\right\|_2
\leq
y_i+\dfrac{1}{4}
$$
If we square to the both side, we can get that
\begin{align*}
    \|P_i^{\frac{1}{2}}x\|_2^2+(y_i-\dfrac{1}{4})^2 &\leq (y_i+\dfrac{1}{4})^2\\
    x^TP_ix+y_i^2-\dfrac{1}{2}y_i+\dfrac{1}{16} &\leq y_i^2+\dfrac{1}{2}y_i+\dfrac{1}{16} \\
    x^TP_ix &\leq y_i
\end{align*}
So the program can be rewriten as
\begin{align*}
    \min_{\mathbf{x},\mathbf{y},u,t}\ \ \ \ \ & u+t+q^T \mathbf{x}+r \\
    \text {subject to }\ \ \ \ \ & A \mathbf{x} \preceq b \\
    & \dfrac{1}{2}\|y\|_2 \leq t \\
    & \left\|
    \left[\begin{array}{l}
    P_0^{\frac{1}{2}}x \\
    u-\dfrac{1}{2}
    \end{array}\right]\right\|_2
    \leq
    u+\dfrac{1}{2} \\
    & \left\|
    \left[\begin{array}{l}
    P_i^{\frac{1}{2}}x \\
    y_i-\dfrac{1}{4}
    \end{array}\right]\right\|_2
    \leq
    y_i+\dfrac{1}{4}
\end{align*}

So above all, the problem is a SOCP.\\

\newpage